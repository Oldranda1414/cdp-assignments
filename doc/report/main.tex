\documentclass[12pt, a4paper]{report}
\usepackage[pdftex]{graphicx} %for embedding images
\usepackage[english,italian]{babel}
\usepackage{url} %for proper url entries
% \usepackage[bookmarks, colorlinks=false, pdfborder={0 0 0}, pdftitle={<pdf title here>}, pdfauthor={<author's name here>}, pdfsubject={<subject here>}, pdfkeywords={<keywords here>}]{hyperref} %for creating links in the pdf version and other additional pdf attributes, no effect on the printed document
%\usepackage[final]{pdfpages} %for embedding another pdf, remove if not required

\begin{document}
\renewcommand\bibname{References} %Renames "Bibliography" to "References" on ref page


\begin{titlepage}

\begin{center}

\Large \textbf {Programmazione Concorrente e Distribuita - Assigment 01}\\%\\[0.5in]
\vspace{1em}%
\vfill
Leonardo Randacio


Filippo Gurioli


Andrea Biagini
\vspace{1em}
\vfill
{\bf Università di Bologna \\ Scienze e Ingegneria Informatiche}\\[0.5in]

       
\vfill
\today

\end{center}

\end{titlepage}


\tableofcontents
\listoffigures
\listoftables

\newpage
\pagenumbering{arabic} %reset numbering to normal for the main content

\chapter{Analysis}
The goal is to create a concurrent agent-based simulation.

An agent-based simulation or model is a computational modeling
 technique used to simulate complex systems by representing individual
 entities, known as agents, and their interactions within an environment.
 The goal of the simulation is to observe the evolution of the states of the
 environment and the agents in each discrete step.

Agents beheviour for a single step can be described in 3 phases:
\begin{itemize}
   \item sense phase: the agent acquires data from the environment
   \item decide phase: the agent determines the next action
   \item act phase: the action determined is executed on the environment
\end{itemize}

\section{Task Decomposition}
Each agent's step can compose a single task, which can be divided into 3 subtasks,
 one for each phase of the step. Since the sense and decide phase are closely
 linked they can be considered as a single task.

This means that for a given step there will be 2 tasks:
\begin{itemize}
    \item sense-decide
    \item act
\end{itemize}

The sense-decide task uses data from the environment and decides
 the next action a certain agent should executed. This task can
 be executed in a parallel manner as no data is modified, exept
 from the state of the agent itself, which would be modified only
 one time every step and only by its personal sense-decide task.

The act task must be serialized as a given simulation must register
 the same result as a sequential simulation would give.

\section{Data Decomposition}
The environment can be subdevided in agents states, based
 on the fact that a given agents's state will be modified
 by only one sense-decide task for a given step

\chapter{Design}

% \input{./prob-definition.tex} %objective changed to problem definition
\bibliographystyle{plain}
\bibliography{References}

\end{document}